\nonstopmode{}
\documentclass[a4paper]{book}
\usepackage[times,inconsolata,hyper]{Rd}
\usepackage{makeidx}
\usepackage[utf8,latin1]{inputenc}
% \usepackage{graphicx} % @USE GRAPHICX@
\makeindex{}
\begin{document}
\chapter*{}
\begin{center}
{\textbf{\huge Package `tp368'}}
\par\bigskip{\large \today}
\end{center}
\begin{description}
\raggedright{}
\item[Type]\AsIs{Package}
\item[Title]\AsIs{What the package does (short line)}
\item[Version]\AsIs{1.0}
\item[Date]\AsIs{2018-01-29}
\item[Author]\AsIs{qingye yuan}
\item[Maintainer]\AsIs{Developer }\email{Developer@some.domain.net}\AsIs{}
\item[Description]\AsIs{More about what it does (maybe more than one line)}
\item[License]\AsIs{What license is it under?}
\end{description}
\Rdcontents{\R{} topics documented:}
\inputencoding{utf8}
\HeaderA{tp368-package}{What the package does (short line)}{tp368.Rdash.package}
\aliasA{tp368}{tp368-package}{tp368}
\keyword{package}{tp368-package}
%
\begin{Description}\relax
More about what it does (maybe more than one line)
\end{Description}
%
\begin{Details}\relax

The DESCRIPTION file:
This package was not yet installed at build time.\\{}

Index:  This package was not yet installed at build time.\\{}
\textasciitilde{}\textasciitilde{} An overview of how to use the package, including the most important functions \textasciitilde{}\textasciitilde{}
\end{Details}
%
\begin{Author}\relax
qingye yuan

Maintainer: Developer <Developer@some.domain.net>
\end{Author}
%
\begin{References}\relax
\textasciitilde{}\textasciitilde{} Literature or other references for background information \textasciitilde{}\textasciitilde{}
\end{References}
%
\begin{SeeAlso}\relax
\textasciitilde{}\textasciitilde{} Optional links to other man pages, e.g. \textasciitilde{}\textasciitilde{}
\textasciitilde{}\textasciitilde{} \code{\LinkA{<pkg>}{<pkg>}} \textasciitilde{}\textasciitilde{}
\end{SeeAlso}
%
\begin{Examples}
\begin{ExampleCode}
~~ simple examples of the most important functions ~~
\end{ExampleCode}
\end{Examples}
\inputencoding{utf8}
\HeaderA{preprocess\_train\_data}{preprocess\_train\_data}{preprocess.Rul.train.Rul.data}
\keyword{\textbackslash{}textasciitilde{}kwd1}{preprocess\_train\_data}
\keyword{\textbackslash{}textasciitilde{}kwd2}{preprocess\_train\_data}
%
\begin{Description}\relax

This is a good packages
\end{Description}
%
\begin{Usage}
\begin{verbatim}
preprocess_train_data(original_set)
\end{verbatim}
\end{Usage}
%
\begin{Arguments}
\begin{ldescription}
\item[\code{original\_set}] 


\end{ldescription}
\end{Arguments}
%
\begin{Examples}
\begin{ExampleCode}
##---- Should be DIRECTLY executable !! ----
##-- ==>  Define data, use random,
##--	or do  help(data=index)  for the standard data sets.

## The function is currently defined as
function (original_set) 
{
    train_set = original_set[, pickedColums]
    train_base = train_set[complete.cases(train_set), ]
    train_base = as.data.frame(train_base)
    train_base$sign = as.factor(train_base$sign)
    return(train_base)
  }
\end{ExampleCode}
\end{Examples}
\printindex{}
\end{document}
